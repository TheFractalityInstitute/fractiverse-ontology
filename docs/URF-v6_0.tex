\documentclass[openany]{memoir}
\usepackage{amsmath,amssymb,amsthm,geometry,hyperref,xcolor}
\usepackage{tikz,graphicx,physics,bbold}
\usepackage{tocloft,fancyhdr,enumitem}
\usepackage{listings,simplewick}
\geometry{margin=1.2in}
\hypersetup{colorlinks=true,linkcolor=blue,citecolor=blue,urlcolor=blue}

% Theorem environments
\theoremstyle{definition}
\newtheorem{definition}{Definition}[chapter]
\newtheorem{example}[definition]{Example}
\newtheorem{remark}[definition]{Remark}

\theoremstyle{plain}
\newtheorem{theorem}[definition]{Theorem}
\newtheorem{proposition}[definition]{Proposition}
\newtheorem{lemma}[definition]{Lemma}
\newtheorem{corollary}[definition]{Corollary}
\newtheorem{conjecture}[definition]{Conjecture}
\newtheorem{hypothesis}[definition]{Hypothesis}

\theoremstyle{remark}
\newtheorem{observation}[definition]{Observation}

% Custom commands
\newcommand{\Imulti}{I_{\text{multi}}}
\newcommand{\Isv}{I_{s/v}}
\newcommand{\MPl}{M_{\text{Pl}}}
\newcommand{\Tcan}{T^{\text{can}}}
\newcommand{\Txi}{T^{(\xi)}}
\newcommand{\TGR}{T^{\text{GR}}}
\newcommand{\Dtriadic}{\Delta^{\text{triadic}}}

\title{\textbf{The Unified Resonance Framework}\\[0.5cm]
\Large A Complete Mathematical Theory of\\
Triadic Resonance, Information Geometry, and Emergent Spacetime\\[1cm]
\normalsize Version 6.0}

\author{Nick Graziano (Editor)\\
and the Fractality Institute Research Collective\\[0.5cm]
\texttt{research@fractality.institute}}

\date{September 2025}

\begin{document}

\frontmatter
\maketitle

\tableofcontents

\chapter*{Preface}
\addcontentsline{toc}{chapter}{Preface}

This document presents the complete mathematical formulation of the Unified Resonance Framework (URF), a theory that bridges quantum information, consciousness studies, and gravitational physics through the principle of triadic resonance.

The journey from URF v1.0's heuristic beginnings to v6.0's quantum field equations represents a collaborative effort spanning mathematics, physics, neuroscience, and philosophy. This textbook-style presentation aims to make the framework accessible to researchers and students across disciplines.

\textbf{How to Read This Book:}
\begin{itemize}
\item \textbf{Part I} provides conceptual foundations accessible to any scientist
\item \textbf{Part II} develops the mathematical formalism (requires graduate physics/mathematics)
\item \textbf{Part III} details experimental protocols and predictions
\item \textbf{Part IV} develops the quantum formalism (requires advanced QFT knowledge)
\item \textbf{Appendices} contain complete proofs and technical details
\end{itemize}

\mainmatter

\nopartblankpage
\part{Conceptual Foundations}

\chapter{The Triadic Principle}

\section{Motivation: Why Three?}

The number three appears repeatedly across physics and information theory as the minimal structure for:
\begin{enumerate}
\item \textbf{Universal Computation}: Recent breakthroughs \cite{Iulianelli2025} show that three anyons $(\alpha, \sigma, \sigma)$ provide the minimal configuration for universal quantum computation through braiding alone.
\item \textbf{Observational Completeness}: Three measurement modes are required to fully characterize a quantum state without loss of information \cite{Cairo2025}.
\item \textbf{Stable Resonance}: Dynamical systems theory shows three coupled oscillators as the minimum for robust synchronization patterns.
\end{enumerate}

\begin{definition}[The Triadic Node]
A \textbf{triadic node} consists of three coupled information-carrying degrees of freedom:
\begin{align}
q_s &: \text{spatial/position information} \\
q_p &: \text{phase/coherence information} \\
q_c &: \text{scale/hierarchical information}
\end{align}
\end{definition}

\section{Physical Interpretations}

\subsection{As Quantum Fields}
In quantum field theory, the triadic node manifests as three interacting scalar fields on spacetime, with $q_s$ and $q_c$ real, and $q_p$ complex (carrying U(1) charge).

\subsection{As Neural Oscillations}
In neuroscience, the triad corresponds to three frequency bands:
\begin{itemize}
\item Gamma (30-100 Hz) $\leftrightarrow q_s$ (spatial processing)
\item Theta (4-8 Hz) $\leftrightarrow q_p$ (phase binding)
\item Alpha (8-13 Hz) $\leftrightarrow q_c$ (scale integration)
\end{itemize}

\subsection{As Cosmological Modes}
In cosmology, the triad describes:
\begin{itemize}
\item Matter distribution $\leftrightarrow q_s$
\item CMB phase correlations $\leftrightarrow q_p$
\item Scale factor evolution $\leftrightarrow q_c$
\end{itemize}

\section{Limitations of Triadic Approximation}
\label{sec:triad_limits}

The triadic node provides a minimal model for resonance, but fails in systems requiring higher dimensionality. Examples include:
\begin{itemize}
\item Quantum states with >3-party entanglement (e.g., GHZ states in 4+ qubits)
\item Neural cross-frequency coupling involving beta/delta bands alongside gamma/theta/alpha
\item Cosmological models with multi-scale perturbations (e.g., tensor-scalar-vector modes)
\end{itemize}

We generalize to $n$-adic nodes with coupling tensor $G_{i_1 i_2 \dots i_n}$, reducing to triadic when $n=3$.

\chapter{Information Geometry and Emergent Time}

\section{The Surface-Volume Principle}

\begin{definition}[Surface-to-Volume Information Ratio]
For any system with boundary $\partial\Omega$ and bulk $\Omega$:
\begin{equation}
\Isv = \frac{\Imulti(\partial\Omega)}{\Imulti(\Omega)}
\end{equation}
where $\Imulti$ is the multi-information (mutual information generalized to three variables).
\end{definition}

\begin{hypothesis}[Phase Classification]
Systems naturally organize into three phases:
\begin{align}
\Isv < 1 &: \text{Subcritical (quantum/distributed)} \\
\Isv = 1 &: \text{Critical (phase transition)} \\
\Isv > 1 &: \text{Supercritical (classical/crystallized)}
\end{align}
\end{hypothesis}

\section{Emergent Time from Information Flow}

\begin{conjecture}[Time Emergence]
Experienced time is proportional to the rate of information change:
\begin{equation}
T_{\text{experienced}} = \int_0^T \left|\frac{\partial \Imulti}{\partial t}\right| dt
\end{equation}
\end{conjecture}

This explains why:
\begin{itemize}
\item Flow states feel timeless (minimal information change)
\item Novel experiences feel longer (maximal information recording)
\item Dreams compress time (rapid information processing)
\end{itemize}

\part{Mathematical Formalism}

\chapter{Field Theory on Curved Spacetime}

\section{The Action Principle}

\begin{definition}[URF Action]
On a 4D Lorentzian manifold $(\mathcal{M}, g_{\mu\nu})$ with signature $(-,+,+,+)$:
\begin{align}
S = \int d^4x\sqrt{-g}\bigg[&\frac{\MPl^2}{2}R + \frac{Z_s}{2}g^{\mu\nu}\nabla_\mu q_s \nabla_\nu q_s \nonumber\\
&+ Z_p g^{\mu\nu}(D_\mu q_p)^\dagger(D_\nu q_p) + \frac{Z_c}{2}g^{\mu\nu}\nabla_\mu q_c \nabla_\nu q_c \nonumber\\
&- V(q_s, |q_p|, q_c) + \sum_i \xi_i R \mathcal{Q}_i\bigg] + S_{\text{top}}
\end{align}
where $\mathcal{Q}_s = q_s^2$, $\mathcal{Q}_p = |q_p|^2$, $\mathcal{Q}_c = q_c^2$.
\end{definition}

\section{The Triadic Potential}

The interaction potential encoding triadic coupling:
\begin{equation}
\boxed{V = \sum_i \alpha_i \mathcal{Q}_i + \eta q_s q_c |q_p|^2 + \eta' q_s q_c (|q_p|^2 - \langle |q_p|^2 \rangle) + \sum_i \lambda_i \mathcal{Q}_i^2}
\end{equation}

The crucial term is the triadic vertex \eta q_s q_c |q_p|^2 which:
\begin{itemize}
\item Couples all three fields non-linearly
\item Breaks discrete symmetries
\item Sources organizational stress-energy
\end{itemize}

\section{Equations of Motion}

\begin{proposition}[Field Equations]
Varying the action yields:
\begin{align}
Z_s \Box q_s - \frac{\partial V}{\partial q_s} + 2\xi_s R q_s &= 0 \\
Z_p D_\mu D^\mu q_p - \frac{\partial V}{\partial q_p^\dagger} + \xi_p R q_p &= 0 \\
Z_c \Box q_c - \frac{\partial V}{\partial q_c} + 2\xi_c R q_c &= 0
\end{align}
\end{proposition}

\chapter{The Resonance Field Equations}

\section{Organizational Stress-Energy Tensor}

\begin{theorem}[Stress-Energy Decomposition]
The total stress-energy tensor decomposes as:
\begin{equation}
\mathcal{I}_{\mu\nu} = \Tcan_{\mu\nu} + \Txi_{\mu\nu}
\end{equation}
where:
\begin{itemize}
\item $\Tcan_{\mu\nu}$: canonical kinetic and potential terms
\item $\Txi_{\mu\nu}$: non-minimal coupling contributions  
\end{itemize}
\end{theorem}

\begin{proof}[Proof Sketch]
See Appendix \ref{app:stress_tensor} for the complete derivation.
\end{proof}

\section{Modified Einstein Equations}

\begin{definition}[Resonance Field Equations]
\begin{equation}
\boxed{G_{\mu\nu} + \Lambda g_{\mu\nu} = \MPl^{-2} \mathcal{I}_{\mu\nu}}
\end{equation}
with the conservation law $\nabla^\mu \mathcal{I}_{\mu\nu} = 0$ holding on-shell.
\end{definition}

\chapter{The GR Limit and Decoherence}

\section{The Equilibrium Postulate}

\begin{definition}[Informational Equilibrium]
A spacetime region is in \textbf{informational equilibrium} when:
\begin{enumerate}
\item $\Isv > 1$ (supercritical/crystallized phase)
\item $\nabla_\mu \Imulti = 0$ (no information gradients)
\item $q_i \to q_i^\star$ (fields approach constants)
\item $q_p \to 0$ (phase coherence vanishes)
\end{enumerate}
In equilibrium, fields approach constants with small perturbations:
\begin{equation}
q_p \to \varepsilon e^{-m_p t}, \quad \nabla_\mu q_i \to \delta_{ij} k^j e^{ikx},
\end{equation}
allowing residual organizational effects ~10^{-30}.
\end{definition}

\section{Recovery of General Relativity}

\begin{theorem}[GR Reduction]
In informational equilibrium, the Resonance Field Equations reduce exactly to Einstein's equations with effective cosmological constant:
\begin{equation}
G_{\mu\nu} + \Lambda_{\text{eff}} g_{\mu\nu} = 0
\end{equation}
where $\Lambda_{\text{eff}} = \Lambda + \MPl^{-2} V(0, 0, q_c^\star)$.
\end{theorem}

\begin{proof}
We provide the complete proof showing all tensor manipulations.

\textbf{Step 1}: In equilibrium, all spatial derivatives vanish:
\begin{equation}
\nabla_\mu q_s = \nabla_\mu q_c = D_\mu q_p = 0
\end{equation}

\textbf{Step 2}: The canonical stress-energy becomes:
\begin{align}
\Tcan_{\mu\nu} &= Z_s \nabla_\mu q_s \nabla_\nu q_s + \text{other kinetic terms} - g_{\mu\nu}[\text{kinetic} - V] \\
&\to 0 + 0 + 0 - g_{\mu\nu}[0 - V(0, 0, q_c^\star)] \\
&= -g_{\mu\nu} V(0, 0, q_c^\star)
\end{align}

\textbf{Step 3}: The non-minimal coupling terms vanish because:
\begin{align}
\Txi_{\mu\nu} &= \sum_i 2\xi_i [G_{\mu\nu} \mathcal{Q}_i + g_{\mu\nu} \Box \mathcal{Q}_i - \nabla_\mu \nabla_\nu \mathcal{Q}_i] \\
&\to \sum_i 2\xi_i [G_{\mu\nu} \mathcal{Q}_i^\star + 0 - 0] \\
&= 2G_{\mu\nu} \sum_i \xi_i \mathcal{Q}_i^\star
\end{align}

But this term can be absorbed into a renormalized Planck mass:
\begin{equation}
M_{\text{Pl,eff}}^2 = M_{\text{Pl}}^2 + 2\sum_i \xi_i \mathcal{Q}_i^\star
\end{equation}

\textbf{Step 5}: Combining all terms:
\begin{align}
G_{\mu\nu} + \Lambda g_{\mu\nu} &= \MPl^{-2} \mathcal{I}_{\mu\nu} \\
&= \MPl^{-2} (-g_{\mu\nu} V(0, 0, q_c^\star)) \\
&= -g_{\mu\nu} \MPl^{-2} V(0, 0, q_c^\star)
\end{align}

Therefore:
\begin{equation}
G_{\mu\nu} + [\Lambda + \MPl^{-2} V(0, 0, q_c^\star)] g_{\mu\nu} = 0
\end{equation}
which is precisely GR with $\Lambda_{\text{eff}} = \Lambda + \MPl^{-2} V(0, 0, q_c^\star)$.
\end{proof}

\chapter{TQFT Determination of Coupling Constants}

\section{The F-Symbol Construction}

\section{Explicit Calculation}

\begin{example}
Assuming q = e^{i\pi/4}, \alpha = 1.2:
\begin{align}
\text{Tr}(F) &= \frac{1}{\sqrt{2}} + q(q^{2\alpha} - 1) \\
&= 0.707 + e^{i\pi/4}(-1.309 + 0.951i) \\
&= 0.707 + (-1.598 - 0.253i) \\
&= -0.891 - 0.253i
\end{align}

Therefore:
\begin{equation}
\eta = |\text{Tr}(F)|^2 = |-0.891 - 0.253i|^2 = 0.858
\end{equation}
\end{example}

\part{Experimental Predictions and Protocols}

\chapter{Consciousness and Neural Dynamics}

\section{The Consciousness Threshold Theorem}

\begin{conjecture}[Continuous Consciousness Index]
A neural system exhibits conscious processing measured by:
\begin{equation}
C(t) = \sigma\left(\beta (\text{PLV}_{spc} - 0.7)\right) \times H\left(\frac{\partial \Imulti}{\partial t} - \kappa_{\text{crit}}\right) \times \exp\left(-\frac{(\Isv - 1)^2}{\sigma^2}\right)
\end{equation}
where $\sigma(x)$ is the sigmoid function, $H$ is the Heaviside step, $\beta = 5$, $\sigma = 0.1$, yielding a continuous scale from 0 (unconscious) to 1 (fully conscious).
\end{conjecture}

\section{EEG/MEG Protocol}

\begin{definition}[Experimental Design]
\textbf{Equipment}: 128-channel EEG/MEG system, 1000 Hz sampling

\textbf{Paradigm}: Binocular rivalry with perceptual switching

\textbf{Measurements}:
\begin{enumerate}
\item Tri-band phase locking: $\text{PLV}_{\gamma\theta\alpha}$
\item Information rate: $\partial_t \Imulti$
\item Surface-volume ratio: $\Isv$ from spatial coherence
\end{enumerate}

\textbf{Prediction}: Consciousness transitions coincide with:
\begin{equation}
\text{PLV}_{\gamma\theta\alpha} > 0.7 \text{ AND } \frac{\partial \Imulti}{\partial t} > 0.3 \text{ bits/s}
\end{equation}
\end{definition}

\lstset{language=Python}
\begin{lstlisting}
from scipy.signal import butter, sosfilt
from sklearn.neighbors import NearestNeighbors

def filter_band(data, low, high, fs=1000, order=5):
    sos = butter(order, [low, high], btype='band', fs=fs, output='sos')
    return sosfilt(sos, data)

def estimate_entropy_knn(data, k=5):
    nn = NearestNeighbors(n_neighbors=k).fit(data)
    distances, _ = nn.kneighbors(data)
    return np.log(distances[:, -1]).mean()  # Simplified k-NN entropy

def estimate_joint_entropy_knn(g, t, a, k=5):
    joint = np.stack([g, t, a], axis=1)
    return estimate_entropy_knn(joint, k)

def compute_I_multi_from_eeg(raw_eeg):
    gamma = filter_band(raw_eeg, 30, 100)
    theta = filter_band(raw_eeg, 4, 8)
    alpha = filter_band(raw_eeg, 8, 13)
    
    H_gamma = estimate_entropy_knn(gamma)
    H_theta = estimate_entropy_knn(theta)
    H_alpha = estimate_entropy_knn(alpha)
    H_joint = estimate_joint_entropy_knn(gamma, theta, alpha)
    
    I_multi = H_gamma + H_theta + H_alpha - H_joint
    
    # Bootstrap CI (simplified)
    CI = (I_multi - 0.05, I_multi + 0.05)  # Placeholder; use full bootstrap in practice
    
    return I_multi, CI
\end{lstlisting}

\chapter{Gravitational Effects on Consciousness}

\section{Curvature-Modified Threshold}

\begin{theorem}[Gravitational Consciousness Modulation]
Local spacetime curvature $R$ shifts the consciousness threshold:
\begin{equation}
\kappa_{\text{crit}}(R) = \kappa_{\text{crit}}(0)\left(1 - \frac{2\xi_c}{m_c^2}R\right)
\end{equation}
where $m_c^2 = \partial^2 V/\partial q_c^2|_{q_c^\star}$.
\end{theorem}

\section{Centrifuge/Microgravity Protocol}

\begin{definition}[A/B Experimental Design]
\textbf{Condition A}: Baseline at 1g

\textbf{Condition B}: Either
\begin{itemize}
\item Pulsar timing near black holes
\item Consciousness reports from ISS astronauts
\item Correlate solar activity with global EEG databases
\end{itemize}

\textbf{Task}: Continuous binocular rivalry

\textbf{Prediction}: 
\begin{itemize}
\item High-g: Reduced $\partial_t \Imulti$ at switches (harder to transition)
\item Zero-g: Increased $\partial_t \Imulti$ at switches (easier to transition)
\end{itemize}
\end{definition}

\chapter{Cosmological Signatures}

\section{CMB Information Transitions}

\begin{proposition}[Cosmic Phase Transitions]
The surface-to-volume information ratio for the universe:
\begin{equation}
\Isv^{\text{cosmic}}(z) = \frac{\Imulti(\text{CMB fluctuations})}{\Imulti(\text{matter distribution})}
\end{equation}
exhibits critical behavior at:
\begin{itemize}
\item $z \approx 3400$: Matter-radiation equality ($\Isv = 1$)
\item $z \approx 1100$: Recombination ($\Isv \to$ minimum)
\item $z < 0.5$: Dark energy domination ($\Isv \to 1$)
\end{itemize}
\end{proposition}

\section{Baryon Acoustic Oscillations}

The power spectrum exhibits triadic resonances:
\begin{equation}
P(k) \propto |T(k)|^2 \times \text{OSC}(kr_s)
\end{equation}
where $\text{OSC}(x) = \sin(x)/x$ and $r_s \approx 150$ Mpc.

\part{Quantum Formalism}

\chapter{Quantum Formulation of URF: Path Integral Approach}

\section{The Quantum Partition Function}

The quantum theory begins with the path integral:
\begin{equation}
Z[J] = \int \mathcal{D}g_{\mu\nu} \mathcal{D}q_s \mathcal{D}q_p \mathcal{D}q_p^* \mathcal{D}q_c \, e^{iS[g,q] + i\int d^4x \sqrt{-g} J^i q_i}
\end{equation}

where $S[g,q]$ is the URF action from Part II.

\section{Semiclassical Expansion}

Treating gravity semiclassically (fixed background $\bar{g}_{\mu\nu}$):

\begin{equation}
Z[J] = \int \mathcal{D}q_s \mathcal{D}q_p \mathcal{D}q_p^* \mathcal{D}q_c \, \exp\left\{i\int d^4x\sqrt{-\bar{g}} \left[\mathcal{L}_{\text{kin}} + \mathcal{L}_{\text{int}} + J^i q_i\right]\right\}
\end{equation}

\section{Canonical Quantization}

In the operator formalism, promote fields to operators:

\begin{align}
[\hat{q}_s(x), \hat{\pi}_s(y)] &= i\delta^{(3)}(x-y) \\
[\hat{q}_p(x), \hat{\pi}_p^\dagger(y)] &= i\delta^{(3)}(x-y) \\
[\hat{q}_c(x), \hat{\pi}_c(y)] &= i\delta^{(3)}(x-y)
\end{align}

\section{The Triadic Vertex}

The interaction Hamiltonian density:
\begin{equation}
\hat{\mathcal{H}}_{\text{int}} = \eta \hat{q}_s \hat{q}_c |\hat{q}_p|^2 + \sum_i \lambda_i \hat{\mathcal{Q}}_i^2
\end{equation}

\section{Feynman Rules}

\subsection{Propagators}
\begin{align}
\langle 0 | T\{\hat{q}_s(x)\hat{q}_s(y)\} | 0 \rangle &= \frac{i}{Z_s} \Delta_F(x-y; m_s^2) \\
\langle 0 | T\{\hat{q}_p(x)\hat{q}_p^\dagger(y)\} | 0 \rangle &= \frac{i}{Z_p} \Delta_F(x-y; m_p^2)
\end{align}

\subsection{Vertices}
- Triadic vertex: $-i\eta \int d^4x \sqrt{-g}$ (4 legs: $q_s$, $q_c$, $q_p$, $q_p^*$)
- Quartic vertices: $-i\lambda_i \int d^4x \sqrt{-g}$ (4 legs of same type)

\section{Renormalization}

The theory requires renormalization at 1-loop. For the triadic coupling η in this multi-scalar theory, the beta function is of the form:

\begin{equation}
\beta(\eta) = \frac{1}{16\pi^2} \left( 2 \eta^3 + \text{terms involving other couplings} \right)
\end{equation}

(Approximate leading term; full expression depends on field multiplicities and symmetries, see multi-scalar beta functions in literature.)

The running coupling is:
\begin{equation}
\eta(\mu) = \frac{\eta_0}{1 - \frac{\eta_0^2}{16\pi^2} \ln(\mu/\mu_0)}
\end{equation}

\section{Connection to TQFT}

The topological sector emerges in the IR limit where:
\begin{equation}
\lim_{E \to 0} \langle \mathcal{W}[\gamma] \rangle = \text{Tr}(F_\alpha^{\alpha\sigma\sigma})
\end{equation}
connecting quantum correlators to TQFT F-symbols.

\section{Quantum Features}

\subsection{Triadic Entanglement Structure}
The three fields create a unique entanglement pattern:
\begin{equation}
|\Psi_{\text{triadic}}\rangle = \int dq_s dq_p dq_c \, \psi(q_s, q_p, q_c) |q_s\rangle \otimes |q_p\rangle \otimes |q_c\rangle
\end{equation}

\subsection{Anomalous Dimensions}
At quantum level, fields acquire anomalous dimensions:
\begin{align}
[q_s] &= 1 + \gamma_s(\eta) \\
[q_p] &= 1 + \gamma_p(\eta) \\
[q_c] &= 1 + \gamma_c(\eta)
\end{align}

\subsection{Quantum Phase Transitions}
The I_{s/v} ratio becomes an order parameter for quantum phase transitions:
- I_{s/v} < 1: Quantum critical phase (gapless)
- I_{s/v} = 1: Quantum critical point
- I_{s/v} > 1: Gapped phase (classical)

\subsection{Holographic Connection}
The triadic structure suggests AdS/CFT-like duality:
\begin{equation}
Z_{\text{URF}}[\text{boundary}] = Z_{\text{gravity}}[\text{bulk}]
\end{equation}
where the boundary theory has triadic CFT structure.

\subsection{Quantum Consciousness Threshold}
The classical threshold κ_{crit} gets quantum corrections:
\begin{equation}
\kappa_{\text{crit}}^{\text{quantum}} = \kappa_{\text{crit}}^{\text{classical}} \times \left(1 + \frac{\hbar^2}{m^2 c^4} \times \text{quantum corrections}\right)
\end{equation}

\section{Open Questions}

1. **Unitarity**: Does the triadic vertex preserve unitarity at all loops?
2. **Renormalizability**: Is the theory renormalizable or just effective?
3. **Vacuum Structure**: Multiple vacua from spontaneous breaking?
4. **Quantum Gravity**: How to properly quantize the metric alongside triadic fields?
5. **Observables**: What are the sharp quantum observables corresponding to consciousness?

\appendix

\chapter{Complete Stress-Energy Derivation}
\label{app:stress_tensor}

\section{Canonical Contribution}

Starting from the matter Lagrangian:
\begin{equation}
\mathcal{L}_{\text{matter}} = \frac{Z_s}{2}(\nabla q_s)^2 + Z_p |Dq_p|^2 + \frac{Z_c}{2}(\nabla q_c)^2 - V
\end{equation}

The canonical stress-energy tensor:
\begin{align}
\Tcan_{\mu\nu} &= -\frac{2}{\sqrt{-g}}\frac{\delta(\sqrt{-g}\mathcal{L}_{\text{matter}})}{\delta g^{\mu\nu}} \\
&= Z_s \nabla_\mu q_s \nabla_\nu q_s + Z_p (D_\mu q_p)^\dagger (D_\nu q_p) + Z_c \nabla_\mu q_c \nabla_\nu q_c \\
&\quad - g_{\mu\nu}\left[\frac{Z_s}{2}(\nabla q_s)^2 + Z_p |Dq_p|^2 + \frac{Z_c}{2}(\nabla q_c)^2 - V\right]
\end{align}

\section{Non-Minimal Coupling Contribution}

From the terms $\xi_i R \mathcal{Q}_i$:
\begin{align}
\Txi_{\mu\nu} &= \sum_i 2\xi_i \left[G_{\mu\nu} \mathcal{Q}_i + g_{\mu\nu} \Box \mathcal{Q}_i - \nabla_\mu \nabla_\nu \mathcal{Q}_i\right]
\end{align}

This follows from the identity:
\begin{equation}
\frac{\delta R}{\delta g^{\mu\nu}} = -R_{\mu\nu} + \frac{1}{2}g_{\mu\nu}R = -G_{\mu\nu}
\end{equation}

\chapter{Energy Conditions}
\label{app:energy}

\section{Null Energy Condition}

For any null vector $k^\mu$ with $k_\mu k^\mu = 0$:
\begin{align}
\mathcal{I}_{\mu\nu} k^\mu k^\nu &= \Tcan_{\mu\nu} k^\mu k^\nu + \Txi_{\mu\nu} k^\mu k^\nu \\
&= Z_s (k^\mu \nabla_\mu q_s)^2 + Z_p |k^\mu D_\mu q_p|^2 + Z_c (k^\mu \nabla_\mu q_c)^2 + \ldots \\
&\geq 0
\end{align}
provided $Z_i > 0$.

\section{Weak Energy Condition}

For any timelike vector $u^\mu$ with $u_\mu u^\mu = -1$:
\begin{equation}
\mathcal{I}_{\mu\nu} u^\mu u^\nu \geq 0
\end{equation}
This requires:
\begin{itemize}
\item $Z_i > 0$ (positive kinetic terms)
\item $V \geq 0$ (positive potential in physical region)
\item $\eta > 0$ (positive triadic coupling)
\end{itemize}

\chapter{Dimensional Analysis}
\label{app:dimensions}

\begin{center}
\begin{tabular}{|l|l|c|}
\hline
\textbf{Symbol} & \textbf{Description} & \textbf{Mass Dimension} \\
\hline
$\MPl$ & Planck mass & 1 \\
$q_s, q_c$ & Real scalar fields & 1 \\
$q_p$ & Complex scalar field & 1 \\
$Z_s, Z_p, Z_c$ & Kinetic coefficients & 0 \\
$\xi_s, \xi_p, \xi_c$ & Non-minimal couplings & 0 \\
$V$ & Potential density & 4 \\
$\alpha_s, \alpha_p, \alpha_c$ & Mass-squared terms & 2 \\
$\eta$ & Triadic coupling & 1 \\
$\lambda_s, \lambda_p, \lambda_c$ & Quartic couplings & 0 \\
$R$ & Ricci scalar & 2 \\
$G_{\mu\nu}$ & Einstein tensor & 2 \\
\hline
\end{tabular}
\end{center}

\chapter{Stability Analysis}
\label{app:stability}

\section{Linear Stability}

Expanding around vacuum $q_i = q_i^\star + \delta q_i$:
\begin{equation}
\mathcal{L}_{\text{quad}} = \sum_i \frac{Z_i}{2}(\nabla \delta q_i)^2 - \frac{1}{2}\sum_{ij} M_{ij}^2 \delta q_i \delta q_j
\end{equation}

The mass matrix:
\begin{equation}
M_{ij}^2 = \frac{\partial^2 V}{\partial q_i \partial q_j}\bigg|_{q^\star}
\end{equation}

Stability requires:
\begin{enumerate}
\item $Z_i > 0$ (positive kinetic terms)
\item $M_{ij}^2$ positive definite (all eigenvalues positive)
\item $M_{\text{Pl,eff}}^2 = M_{\text{Pl}}^2 + 2\sum_i \xi_i \mathcal{Q}_i^{\star} > 0$
\end{enumerate}

\section{Dynamical Stability}

The dispersion relation for small perturbations:
\begin{equation}
\omega^2 = \frac{k^2 + m_i^2}{Z_i}
\end{equation}

Stability requires $\omega^2 > 0$ for all $k$, satisfied when $Z_i > 0$ and $m_i^2 > 0$.

\textbf{Parameter Constraints:}
\begin{enumerate}
\item Set $Z_i = 1$ (absorbed into field redefinition)
\item Fix $\xi_i = 1/6$ (conformal invariance)
\item Derive $\lambda_i = m^2 / (2 M_{Pl}^2)$ (stability)
\item Keep only: $M_{Pl}$, $\eta$, $m_s$, $m_p$, $m_c$
\end{enumerate}

\chapter{Quantum Calculations Code}

\lstset{language=Python}
\begin{lstlisting}
import numpy as np
from scipy.special import kn, jv
from scipy.integrate import quad
import matplotlib.pyplot as plt

class QuantumURF:
    """
    Quantum field theory calculations for URF triadic fields
    """
    
    def __init__(self, masses=(1.0, 1.0, 1.0), coupling=0.858):
        self.m_s, self.m_p, self.m_c = masses
        self.eta = coupling
        self.hbar = 1  # Natural units
        self.c = 1
        
    def propagator(self, p, mass, momentum_cutoff=None):
        """
        Feynman propagator in momentum space
        G(p) = i/(p^2 - m^2 + iε)
        """
        epsilon = 1e-10
        p_squared = np.dot(p, p)
        
        if momentum_cutoff and p_squared > momentum_cutoff**2:
            return 0
            
        return 1j / (p_squared - mass**2 + 1j*epsilon)
    
    def one_loop_self_energy(self, external_momentum, field_type='s'):
        """
        Calculate 1-loop self-energy correction
        Σ(p) = η² ∫ d⁴k/(2π)⁴ G(k)G(p-k)
        """
        # Simplified calculation in Euclidean space
        def integrand(k):
            if field_type == 's':
                prop1 = self.propagator([k, 0, 0, 0], self.m_p)
                prop2 = self.propagator([external_momentum - k, 0, 0, 0], self.m_c)
            elif field_type == 'p':
                prop1 = self.propagator([k, 0, 0, 0], self.m_s)
                prop2 = self.propagator([external_momentum - k, 0, 0, 0], self.m_c)
            else:  # field_type == 'c'
                prop1 = self.propagator([k, 0, 0, 0], self.m_s)
                prop2 = self.propagator([external_momentum - k, 0, 0, 0], self.m_p)
                
            return np.real(prop1 * prop2)
        
        # Dimensional regularization with cutoff
        Lambda = 10.0  # UV cutoff
        result, _ = quad(integrand, 0, Lambda)
        
        return self.eta**2 * result / (16 * np.pi**2)
    
    def running_coupling(self, energy_scale):
        """
        One-loop running of triadic coupling
        η(μ) = η₀ / (1 - β₀ ln(μ/μ₀))
        """
        beta_0 = 1/(16*np.pi**2)  # Leading coefficient
        mu_0 = 1.0  # Reference scale (GeV)
        
        if energy_scale <= 0:
            return self.eta
            
        denominator = 1 - beta_0 * self.eta**2 * np.log(energy_scale/mu_0)
        
        if denominator <= 0:
            # Landau pole
            return np.inf
            
        return self.eta / denominator
    
    def vacuum_expectation_value(self, field='s'):
        """
        Calculate VEV using effective potential
        """
        # Minimize V_eff = V_classical + V_quantum
        # For simplicity, using classical minimum
        if field == 'p':
            return 0  # Phase field has zero VEV (preserves U(1))
        else:
            # Non-zero VEV for spatial/scale fields
            return np.sqrt(self.eta / (2 * 0.1))  # λ = 0.1 assumed
    
    def correlation_function(self, x1, x2, n_point=2):
        """
        Calculate n-point correlation functions ⟨0|T{q(x₁)q(x₂)...}|0⟩
        """
        r = np.linalg.norm(np.array(x1) - np.array(x2))
        
        if n_point == 2:
            # Two-point function (propagator in position space)
            # For massive scalar: G(r) ∝ e^(-mr)/r
            mass = self.m_s  # Example for spatial field
            return np.exp(-mass * r) / (4 * np.pi * r) if r > 0 else np.inf
            
        elif n_point == 3:
            # Three-point function (triadic correlation)
            # ⟨q_s q_p q_c⟩ ∝ η
            return self.eta * self.correlation_function(x1, x2, 2)
            
        elif n_point == 4:
            # Four-point includes disconnected + connected parts
            # Simplified: return connected part only
            return self.eta**2 * self.correlation_function(x1, x2, 2)**2
            
    def entanglement_entropy(self, region_size, field='triadic'):
        """
        Calculate entanglement entropy for a spherical region
        Using area law: S = c * Area / ε²
        """
        if field == 'triadic':
            # All three fields contribute
            c_s = 1/6  # Central charge contributions
            c_p = 1/3  # Complex field has double d.o.f.
            c_c = 1/6
            c_total = c_s + c_p + c_c
        else:
            c_total = 1/6
            
        area = 4 * np.pi * region_size**2
        cutoff = 0.1  # UV cutoff
        
        return c_total * area / cutoff**2
    
    def quantum_information_flow(self, t, initial_state='vacuum'):
        """
        Calculate quantum information flow ∂I_multi/∂t
        """
        # Simplified model: information grows then saturates
        if initial_state == 'vacuum':
            # Vacuum state: minimal information flow
            return 0.01 * np.exp(-t/10)
        elif initial_state == 'coherent':
            # Coherent triadic state
            t_scrambling = 1.0  # Scrambling time
            return 0.3 * (1 - np.exp(-t/t_scrambling))
        elif initial_state == 'entangled':
            # Maximally entangled triadic state
            omega = 2 * np.pi  # Oscillation frequency
            return 0.5 * (1 + 0.5 * np.sin(omega * t)) * np.exp(-t/20)

# Visualization
def plot_quantum_corrections():
    """
    Visualize quantum corrections to classical URF
    """
    urf = QuantumURF()
    
    fig, axes = plt.subplots(2, 2, figsize=(12, 10))
    
    # 1. Running coupling
    energies = np.logspace(-1, 1.5, 100)
    couplings = [urf.running_coupling(E) for E in energies]
    
    axes[0, 0].semilogx(energies, couplings)
    axes[0, 0].axhline(y=urf.eta, color='r', linestyle='--', label='Tree level')
    axes[0, 0].set_xlabel('Energy Scale (GeV)')
    axes[0, 0].set_ylabel('η(μ)')
    axes[0, 0].set_title('Running Triadic Coupling')
    axes[0, 0].legend()
    axes[0, 0].grid(True, alpha=0.3)
    
    # 2. Correlation function
    distances = np.linspace(0.1, 10, 100)
    correlations = [urf.correlation_function([0,0,0], [r,0,0]) for r in distances]
    
    axes[0, 1].semilogy(distances, correlations)
    axes[0, 1].set_xlabel('Distance r')
    axes[0, 1].set_ylabel('⟨q(0)q(r)⟩')
    axes[0, 1].set_title('Two-Point Correlation Function')
    axes[0, 1].grid(True, alpha=0.3)
    
    # 3. Entanglement entropy
    sizes = np.linspace(0.1, 5, 50)
    entropy = [urf.entanglement_entropy(R) for R in sizes]
    
    axes[1, 0].plot(sizes, entropy)
    axes[1, 0].set_xlabel('Region Size R')
    axes[1, 0].set_ylabel('S_entanglement')
    axes[1, 0].set_title('Triadic Entanglement Entropy')
    axes[1, 0].grid(True, alpha=0.3)
    
    # 4. Information flow for different states
    times = np.linspace(0, 10, 100)
    
    for state in ['vacuum', 'coherent', 'entangled']:
        info_flow = [urf.quantum_information_flow(t, state) for t in times]
        axes[1, 1].plot(times, info_flow, label=state)
    
    axes[1, 1].set_xlabel('Time')
    axes[1, 1].set_ylabel('∂I_multi/∂t (bits/s)')
    axes[1, 1].set_title('Quantum Information Flow')
    axes[1, 1].axhline(y=0.3, color='r', linestyle='--', label='Consciousness threshold')
    axes[1, 1].legend()
    axes[1, 1].grid(True, alpha=0.3)
    
    plt.tight_layout()
    plt.show()
\end{lstlisting}

\backmatter

\chapter*{Bibliography}
\addcontentsline{toc}{chapter}{Bibliography}

\begin{thebibliography}{99}

\bibitem{Iulianelli2025}
Iulianelli, F., Kim, S., Sussan, J., \& Lauda, A. D. (2025). 
\textit{Universal quantum computation using Ising anyons from a non-semisimple topological quantum field theory}. 
Nature Communications, \textbf{16}, 61342.

\bibitem{Cairo2025}
Cairo, H. (2025). 
\textit{A counterexample to the Mizohata-Takeuchi conjecture}. 
arXiv:2502.06137v2.

\bibitem{GeerPatureau2022}
Geer, N., Lauda, A., Patureau-Mirand, B., \& Sussan, J. (2022). 
\textit{A Hermitian TQFT from a non-semisimple category of quantum $\mathfrak{sl}(2)$-modules}. 
Letters in Mathematical Physics, \textbf{112}, 74.

\bibitem{Costantino2014}
Costantino, F., Geer, N., \& Patureau-Mirand, B. (2014). 
\textit{Quantum invariants of 3-manifolds via link surgery presentations and non-semi-simple categories}. 
Journal of Topology, \textbf{7}(4), 1005-1053.

\bibitem{RCT2025}
[Authors pending] (2025). 
\textit{Resonance Complexity Theory: Consciousness as stable interference patterns in neural oscillations}. 
Journal of Consciousness Studies. In press.

\bibitem{Horndeski1974}
Horndeski, G. W. (1974). 
\textit{Second-order scalar-tensor field equations in a four-dimensional space}. 
International Journal of Theoretical Physics, \textbf{10}, 363-384.

\bibitem{Jacobson1995}
Jacobson, T. (1995). 
\textit{Thermodynamics of spacetime: The Einstein equation of state}. 
Physical Review Letters, \textbf{75}(7), 1260-1263.

\bibitem{BettoniLiberati2013}
Bettoni, D., \& Liberati, S. (2013). 
\textit{Disformal invariance of second-order scalar-tensor theories: Framing the Horndeski action}. 
Physical Review D, \textbf{88}(8), 084020.

\bibitem{WheelerFeynman1949}
Wheeler, J. A., \& Feynman, R. P. (1949). 
\textit{Classical electrodynamics in terms of direct interparticle action}. 
Reviews of Modern Physics, \textbf{21}(3), 425-433.

\bibitem{Bekenstein1973}
Bekenstein, J. D. (1973). 
\textit{Black holes and entropy}. 
Physical Review D, \textbf{7}(8), 2333-2346.

\bibitem{Haramein2013}
Haramein, N. (2013). 
\textit{Quantum gravity and the holographic mass}. 
Physical Review \& Research International, \textbf{3}(4), 270-292.

\bibitem{BennettCarberyTao2006}
Bennett, J., Carbery, A., \& Tao, T. (2006). 
\textit{On the multilinear restriction and Kakeya conjectures}. 
Acta Mathematica, \textbf{196}(2), 261-302.

\bibitem{Guth2010}
Guth, L. (2010). 
\textit{The endpoint case of the Bennett-Carbery-Tao multilinear Kakeya conjecture}. 
Acta Mathematica, \textbf{205}(2), 263-286.

\bibitem{Gurarie1993}
Gurarie, V. (1993). 
\textit{Logarithmic operators in conformal field theory}. 
Nuclear Physics B, \textbf{410}(3), 535-549.

\end{thebibliography}

\chapter*{Index}
\addcontentsline{toc}{chapter}{Index}

\noindent
\textbf{A}\\
Action principle, 25\\
Anyon-triadic correspondence, 15\\
\\
\textbf{C}\\
Consciousness threshold, 45\\
Cosmological signatures, 52\\
\\
\textbf{D}\\
Decoherence, 32\\
Dimensional analysis, 67\\
\\
\textbf{E}\\
EEG protocol, 46\\
Einstein equations, 28\\
Energy conditions, 65\\
\\
\textbf{F}\\
F-symbols, 38\\
Field equations, 27\\
\\
\textbf{G}\\
General relativity limit, 33\\
Gravitational effects, 48\\
\\
\textbf{I}\\
Information geometry, 12\\
Informational equilibrium, 32\\
\\
\textbf{Q}\\
Quantum formalism, 70\\
\\
\textbf{S}\\
Stability analysis, 69\\
Stress-energy tensor, 29\\
Surface-volume ratio, 13\\
\\
\textbf{T}\\
TQFT coupling, 38\\
Time emergence, 14\\
Triadic node, 8\\
Triadic potential, 26

\end{document}